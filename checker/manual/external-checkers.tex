\htmlhr
\chapter{Third-party checkers\label{third-party-checkers}\label{external-checkers}}

The Checker Framework has been used to build other checkers that are not
distributed together with the framework.  This chapter mentions just a few
of them.  They are listed in chonological order; older ones appear first
and newer ones appear last.

They are externally-maintained, so if you have problems or questions, you
should contact their maintainers rather than the Checker Framework
maintainers.

If you want a reference to your checker included in this chapter,
send us a link and a short description.


% Note to maintainers:
% Sections are added to this chapter in chronological order.


\input{typestate-checker}


\section{Units and dimensions checker\label{units-and-dimensions-checker}}

A checker for units and dimensions is available at
\url{http://www.lexspoon.org/expannots/}.

Unlike the Units Checker that is distributed with the Checker Framework
(see Section~\ref{units-checker}), this checker includes dynamic checks and
permits annotation arguments that are Java expressions.  This added
flexibility, however, requires that you use a special version both of the
Checker Framework and of the javac compiler.


\section{Thread locality checker\label{loci-thread-locality-checker}}

Loci~\cite{WrigstadPMZV2009}, a checker for thread locality, is available at
\url{http://www.it.uu.se/research/upmarc/loci/}.
Developer resources are available at the project page
\url{http://java.net/projects/loci/}.

% A paper was publishd in ECOOP 2009, release 0.1 was made in March 2011,
% but as of October 2013 the manual is still listed as "forthcoming".


% In a mail from Amanj Mahmud <amanjpro@gmail.com> on 28.03.2011:

% The plugin name:
% ``Loci: A Pluggable Type Checker for Expressing Thread Locality in
% Java''

% Project homepage: http://www.it.uu.se/research/upmarc/loci

% Project's developer's page: http://java.net/projects/loci


\section{Safety-Critical Java checker\label{safety-critical-java-checker}}

A checker for Safety-Critical Java (SCJ, JSR 302)~\cite{TangPJ2010} is available at
\url{http://sss.cs.purdue.edu/projects/oscj/checker/checker.html}.
Developer resources are available at the project page
\url{https://code.google.com/archive/p/scj-jsr302/}.


% In a mail from Aleš Plšek <aplsek@gmail.com> on 29.03.2011:

% Name: SCJ Checker
% WWW: http://sss.cs.purdue.edu/projects/oscj/checker/checker.html
% Source-Code Repository: http://code.google.com/p/scj-jsr302/

% Description: The SCJ Checker implements verification of a set of
% annotations defined by the Safety-Critical Java standard (JSR-302).
% The checker mainly focuses on proving memory safety of Java programs
% that use a region-based memory management.

% Publications: Static checking of safety critical Java annotations:
% http://portal.acm.org/citation.cfm?doid=1850771.1850792


\section{Generic Universe Types checker\label{gut-checker}}

A checker for Generic Universe Types~\cite{DietlEM2011}, a lightweight ownership type
system, is available from
\url{https://ece.uwaterloo.ca/~wdietl/ownership/}.


\section{EnerJ checker\label{enerj-checker}}

A checker for EnerJ~\cite{SampsonDFGCG2011}, an extension to Java that exposes hardware faults
in a safe, principled manner to save energy with only
slight sacrifices to the quality of service, is available from
\url{http://sampa.cs.washington.edu/research/approximation/enerj.html}.


\section{CheckLT taint checker\label{checklt-checker}}

CheckLT uses taint tracking to detect illegal information flows, such as
unsanitized data that could result in a SQL injection attack.
CheckLT is available from \url{http://checklt.github.io/}.


\section{SPARTA information flow type-checker for Android\label{sparta-checker}}

SPARTA is a security toolset aimed at preventing malware from appearing in
an app store.  SPARTA provides an information-flow type-checker that is
customized to Android but can also be applied to other domains.
The SPARTA toolset is available from
\url{http://types.cs.washington.edu/sparta/}.
The paper
\href{http://homes.cs.washington.edu/~mernst/pubs/infoflow-ccs2014.pdf}{``Collaborative
    verification of information flow for a high-assurance app store''}
  appeared in CCS 2014.


% LocalWords:  SCJ EnerJ CheckLT unsanitized JavaUI
